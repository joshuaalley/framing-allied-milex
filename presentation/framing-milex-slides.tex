\documentclass[12pt]{beamer}


\usetheme[progressbar=frametitle]{metropolis}
\usepackage{appendixnumberbeamer}

\usepackage{booktabs}
\usepackage[scale=2]{ccicons}

%\usepackage{pgfplots}
%\usepgfplotslibrary{dateplot}

\usepackage{xspace}
\newcommand{\themename}{\textbf{\textsc{metropolis}}\xspace}

%\setbeamertemplate{footline} % To remove the footer line in all slides uncomment this line
%\setbeamertemplate{footline}[page number] % To replace the footer line in all slides with a simple slide count uncomment this line

%\setbeamertemplate{navigation symbols}{} % To remove the navigation symbols from the bottom of all slides uncomment this line


\usepackage{graphicx} % Allows including images
\usepackage{grffile}
\usepackage{amsmath}
\usepackage{adjustbox} 
% have to have Mozilla's=Fira Sans} font and XeTeX installed to use full typography.

%----------------------------------------------------------------------------------------
%	TITLE PAGE
%----------------------------------------------------------------------------------------

\title{Collective Action or Exchange?: Framing International Cooperation in Alliance Politics}
\date{November 12, 2020}
\author{Joshua Alley}
\institute{University of Virginia}


\begin{document}

 \maketitle


%----------------------------------------------------------------------------------------
%	PRESENTATION SLIDES
%----------------------------------------------------------------------------------------

%------------------------------------------------
% Here's my point. 
 \begin{frame}[standout]

How do elite frames of allied military spending affect public support for cooperation in alliances? 

 \end{frame}
 

%------------------------------------------------

\begin{frame}{Why Should You Care?}

\begin{figure}[htbp]
		\includegraphics[width=0.95\textwidth]{trump-nato.jpg}
	\label{fig:trump-nato}
\end{figure}


\end{frame}

%------------------------------------------------

\begin{frame}{Outline}

\pause
\begin{enumerate}
\item Argument.
\pause
\item Survey Experiment Design. 
\pause
\item Pretest Results. 
\end{enumerate}


\end{frame}
 

%------------------------------------------------

\section{Argument} 

%-----------------------------------------------

\begin{frame}{Two Frames of International Cooperation}

Can explain the causes and consequences of cooperation with:

\pause 
\begin{enumerate} 
\item \textbf{Collective Action}: Cooperate by contributing to some common/public good. 
\pause 
\item \textbf{Exchange}: Cooperate by trading different goods. 
\end{enumerate}


\end{frame} 

%-----------------------------------------------

\begin{frame}{Framing Spending by Alliance Members}

Foreign policy elites can use either general argument to frame allied military spending. 

\pause 
\begin{enumerate} 
\item \textbf{Collective Action}: Free-riding and disproportionate contributions. 
\pause 
\item \textbf{Exchange}: Trading different foreign policy goods- security for influence. 
\end{enumerate}


\end{frame} 

%-----------------------------------------------

\begin{frame}[standout]

These frames have opposite effects on attitudes towards cooperation in leading and junior institutional members. 

\end{frame} 

%-----------------------------------------------

\begin{frame}{Leading States}

Framing disproportionate defense spending by a large state: 
\pause 
\begin{enumerate} 
\item \textbf{Collective Action}: Decreases support for cooperation: conditional cooperation and exploitation aversion. 
\pause 
\item \textbf{Exchange}: Increases support for cooperation: reciprocity. 
\end{enumerate}


\end{frame} 

%-----------------------------------------------

\begin{frame}{Junior States}

Framing disproportionate defense spending by a large state: 
\pause 
\begin{enumerate} 
\item \textbf{Collective Action}: Increases support for cooperation: conditional cooperation and perceptions of benevolent leadership. 
\pause 
\item \textbf{Exchange}: Decreases support for cooperation: reduces leader legitimacy. 
\end{enumerate}


\end{frame} 



%------------------------------------------------

\section{Experimental Design} 

%-----------------------------------------------

\begin{frame}{Emphasis}

Attitudes towards NATO in the United States and Germany. 210 respondents in each study. Randomly assign neutral, collective action or exchange vignette about NATO and military spending. 

\pause 
\begin{enumerate} 
\item Favorability towards NATO. 
\pause 
\item Support for withdrawal (US) or higher defense spending (Germany).  
\pause
\item Support for military intervention. 
\end{enumerate}


\end{frame} 

%-----------------------------------------------

\begin{frame}{Vignettes}


\begin{enumerate}

\item \textbf{Neutral}: The United States has an important role in the North Atlantic Treaty Organization (NATO). NATO is a military alliance where members promise to support one another in war. According to an expert at the Council on Foreign Relations, a non-partisan think tank, some NATO members spend a smaller share of their resources on the military than the United States. 
\pause
\item \textbf{Collective Action}: adds \textit{because other states make limited contributions to collective security, and count on the United States to carry the load.} 
\pause
\item \textbf{Exchange}: adds \textit{because they support US priorities and interests in international politics in exchange for protection by the United States.}
 
\end{enumerate} 

\end{frame} 


%-----------------------------------------------

\begin{frame}{Response Questions}


\begin{enumerate}

\item \textbf{Favorability}: 1-5 scale from Very Unfavorable to Very Favorable.  
\pause 
\item \textbf{Policy Response}: Yes/No on intervention and military spending. 
\pause 
\item \textbf{Intervention}: Yes/No. 
 
\end{enumerate} 

\end{frame} 



%------------------------------------------------

\section{Pretest Results} 

%-----------------------------------------------


\begin{frame}{United States: Raw Data}

\begin{figure}[htbp]
	\centering
		\includegraphics[height=.95\textheight]{raw-us-pres.png}
\end{figure}


\end{frame}

%-----------------------------------------------

\begin{frame}{United States: Treatment Effects}

% mention regressions
\pause 

\begin{figure}[htbp]
	\centering
		\includegraphics[width=0.95\textwidth]{us-te-pres.png}
\end{figure}


\end{frame}

%-----------------------------------------------


\begin{frame}{Germany: Raw Data}

\begin{figure}[htbp]
	\centering
		\includegraphics[height=.95\textheight]{raw-german-pres.png}
\end{figure}


\end{frame}

%-----------------------------------------------


\begin{frame}{Germany: Treatment Effects}

\begin{figure}[htbp]
	\centering
		\includegraphics[width=0.95\textwidth]{ge-te-pres.png}
\end{figure}


\end{frame}


%-----------------------------------------------


\begin{frame}{Discussion}

These results are weaker than expected:

\pause
\begin{enumerate}
\item Study is underpowered. \pause Also hard to detect heterogeneous effects with this small sample.  
\pause 
\item Pre-treatment of collective action frames in the United States. 
\pause
\item Non-representative pre-test data.
\end{enumerate} 

\end{frame}


%------------------------------------------------

\begin{frame}{Conclusion}

Mixed evidence at best: would love your thoughts on:

\pause
\begin{enumerate}
\item The argument. 
\pause
\item The experimental design. 
\end{enumerate}


\end{frame}

%-----------------------------------------------


%------------------------------------------------

\begin{frame}[standout]

Last, thanks! 

jkalley@virginia.edu

\end{frame}

%-----------------------------------------------

\appendix 

%-----------------------------------------------

\begin{frame}{Demographic Variables/Controls}

\begin{itemize}
\item Partisanship
\item Ideology
\item Foreign Policy Knowledge
\item National Pride 
\item Military Service
\item College Education
\item Age
\end{itemize} 

\end{frame}


%-----------------------------------------------

\begin{frame}{MTurk Sample Concerns}

\begin{itemize}
\item Lots of Democrats
\item Above-average education
\item Above-average foreign policy knowledge. 
\end{itemize} 

\end{frame}


%-----------------------------------------------


\begin{frame}{Partisanship by Group}

\begin{figure}[htbp]
	\centering
		\includegraphics[width=0.95\textwidth]{partisan-group.png}
\end{figure}


\end{frame}

%-----------------------------------------------


\begin{frame}{Mechanism in Neutral Frame}

\begin{figure}[htbp]
	\centering
		\includegraphics[width=0.95\textwidth]{neutral-mech.png}
\end{figure}


\end{frame}


%--------------------------------------------------

\begin{frame}{Map of Favorability}

\begin{figure}[htbp]
\centering
   \includegraphics[width=.95\textwidth]{favor-map.png}
\end{figure}

\end{frame}

%---------------------------------------------------

\begin{frame}{Map of Support for Nonintervention} 

\begin{figure}[htbp]
\centering
   \includegraphics[width = .95\textwidth]{noforce-map.png}
\end{figure}

\end{frame}

%---------------------------------------------------



%----------------------------------------------------------------------------------------

\end{document}
